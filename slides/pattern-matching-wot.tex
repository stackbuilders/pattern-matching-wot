\documentclass{beamer}

\usetheme{default}

\usepackage{polyglossia}

\setdefaultlanguage[variant=usmax]{english}

\usepackage{fancyvrb}

\DefineShortVerb{\|}
\DefineVerbatimEnvironment{code}{Verbatim}{frame=lines}

\title{Pattern Matching}
\subtitle{Wot's... Uh the Deal?}
\author{}
\institute{Stack Builders}
\date{0.1.0}

\begin{document}

\frame{\titlepage}

\begin{frame}[fragile]
  \frametitle{Several Species of Small Furry \texttt{null} functions}

  \begin{code}
null1 :: [a] -> Bool
null1 []    = True
null1 (_:_) = False
  \end{code}
\end{frame}

\begin{frame}[fragile]
  \frametitle{Several Species of Small Furry \texttt{null} functions}

  \begin{code}
null2 :: [a] -> Bool
null2 []    = True
null2 _     = False
  \end{code}
\end{frame}

\begin{frame}[fragile]
  \frametitle{Several Species of Small Furry \texttt{take} functions}

  \begin{code}
take1 :: Int -> [a] -> [a]
take1 n _      | n <= 0 = []
take1 _ []              = []
take1 n (x:xs)          = x : take1 (n - 1) xs
  \end{code}
\end{frame}

\begin{frame}[fragile]
  \frametitle{Several Species of Small Furry \texttt{take} functions}

  \begin{code}
take2 :: Int -> [a] -> [a]
take2 _ []              = []
take2 n _      | n <= 0 = []
take2 n (x:xs)          = x : take2 (n - 1) xs
  \end{code}
\end{frame}

\begin{frame}
  \frametitle{}

  \begin{thebibliography}{}
  \setbeamertemplate{bibliography item}[article]
  \bibitem[]{hudak-peterson-fasel-1999}
    Hudak, Paul, John Peterson, and Joseph H. Fasel (1999).
    \newblock \emph{A Gentle Introduction to Haskell 98}.
    \newblock \url{https://www.haskell.org/tutorial/}
  \setbeamertemplate{bibliography item}[article]
  \bibitem[]{marlow-2010}
    Marlow, Simon, ed. (2010).
    \newblock \emph{Haskell 2010 Language Report}.
  \end{thebibliography}
\end{frame}

\end{document}
